\documentclass{article}
\usepackage[margin=2cm]{geometry}
\usepackage[cmyk]{xcolor}
\usepackage{tikz}
\usepackage{pgf-cmykshadings}
\usepackage{parskip}

\begin{document}

\pgfdeclarehorizontalshading{myshadingA}
  {1cm}{cmyk(0cm)=(0,1,1,0); color(2cm)=(green); color(4cm)=(blue)}
\pgfuseshading{myshadingA}

\pgfdeclarehorizontalshading[mycolor]{myshadingB}
  {1cm}{cmyk(0cm)=(0,1,1,0); color(2cm)=(mycolor)}
\colorlet{mycolor}{green}
\pgfuseshading{myshadingB}
\colorlet{mycolor}{blue}
\pgfuseshading{myshadingB}

\pgfdeclareverticalshading{myshadingC}
  {4cm}{cmyk(0cm)=(0,1,1,0); cmyk(1.5cm)=(1,0,1,0); cmyk(2cm)=(1,1,0,0)}
\pgfuseshading{myshadingC}

\pgfdeclareradialshading{sphere}{\pgfpoint{0.5cm}{0.5cm}}%
  {cmyk(0cm)=(0,0.9,0.9,0);
   cmyk(0.7cm)=(0,0.7,0.7,0.3);
   cmyk(1cm)=(0,0.5,0.5,0.5);
   cmyk(1.05cm)=(0,0,0,0)}
\pgfuseshading{sphere}

\pgfdeclarefunctionalcmykshading{twospots}
    {\pgfpointorigin}{\pgfpoint{4cm}{4cm}}{}{
  % Save coordinates for later
  2 copy
  % magenta component
  % Compute distance from (40bp,45bp), with x doubled
  45 sub dup mul exch
  40 sub dup mul 0.5 mul add sqrt
  % exponential decay
  dup mul neg 1.0005 exch exp 1.0 exch sub
  1.0 exch sub
  % yellow component
  % Compute distance from (70bp,70bp) from stored coordinate, scaled
  3 1 roll
  70 sub dup mul .5 mul exch
  70 sub dup mul add sqrt
  % Decay
  dup mul neg 1.002 exch exp 1.0 exch sub
  1.0 exch sub
  % cyan component
  0.0 3 1 roll
  % black component
  0.0
}
\pgfusecmykshading{twospots}

\pgfdeclarefunctionalcmykshading[mycol]{sweep}{\pgfpoint{-1cm}{-1cm}}
    {\pgfpoint{1cm}{1cm}}{\pgfshadecolortocmyk{mycol}{\mycmyk}}{
  2 copy       % whirl
  % Calculate "safe" atan of position
  2 copy abs exch abs add 0.0001 ge { atan } { pop } ifelse
  3 1 roll
  dup mul exch
  dup mul add sqrt
  30 mul
  add
  sin
  1 add 2 div
  0.0 0.5 1.0
  4 3 roll
  1 exch sub
  \mycmyk      % push mycol
  5 4 roll
  add dup 1.0 ge { pop 1.0 } { } ifelse
}
\colorlet{mycol}{white}%
\pgfusecmykshading{sweep}%
\colorlet{mycol}{red}%
\pgfusecmykshading{sweep}

\pgfdeclarefunctionalcmykshading[col1,col2,col3,col4]{bilinear interpolation}
    {\pgfpointorigin}{\pgfpoint{100bp}{100bp}}
    {
    \pgfshadecolortocmyk{col1}{\first}\pgfshadecolortocmyk{col2}{\second}
    \pgfshadecolortocmyk{col3}{\third}\pgfshadecolortocmyk{col4}{\fourth}
    }{
  100 div exch 100 div 2 copy                        % Calculate y/100 x/100.
  neg 1 add exch neg 1 add                           % Calculate 1-y/100 1-x/100.
  3 1 roll 2 copy exch 5 2 roll 6 copy 6 copy 6 copy % Set up stack.
  \firstcyan mul exch \secondcyan mul add mul        % Process cyan component.
  4 1 roll
  \thirdcyan mul exch \fourthcyan mul add mul
  add
  19 1 roll
  \firstmagenta mul exch \secondmagenta mul add mul  % Process magenta component.
  4 1 roll
  \thirdmagenta mul exch \fourthmagenta mul add mul
  add
  13 1 roll
  \firstyellow mul exch \secondyellow mul add mul    % Process yellow component.
  4 1 roll
  \thirdyellow mul exch \fourthyellow mul add mul
  add
  7 1 roll
  \firstblack mul exch \secondblack mul add mul      % Process black component.
  4 1 roll
  \thirdblack mul exch \fourthblack mul add mul
  add
}
\colorlet{col1}{blue}
\colorlet{col2}{yellow}
\colorlet{col3}{red}
\colorlet{col4}{green}
\pgfusecmykshading{bilinear interpolation}

\begin{pgfpicture}
  \pgfdeclareverticalshading{myshadingD}
    {20pt}{color(0pt)=(red); color(20pt)=(blue)}
  \pgftext[at=\pgfpoint{1cm}{0cm}] {\pgfuseshading{myshadingD}}
  \pgftext[at=\pgfpoint{2cm}{0.5cm}]{\pgfuseshading{myshadingD}}
\end{pgfpicture}

\pgfdeclareverticalshading{myshadingE}{100bp}
  {color(0bp)=(red); color(25bp)=(green); color(75bp)=(blue); color(100bp)=(black)}
\pgfuseshading{myshadingE}
\hskip 1cm
\begin{pgfpicture}
  \pgfpathrectangle{\pgfpointorigin}{\pgfpoint{2cm}{1cm}}
  \pgfshadepath{myshadingE}{0}
  \pgfusepath{stroke}
  \pgfpathrectangle{\pgfpoint{3cm}{0cm}}{\pgfpoint{1cm}{2cm}}
  \pgfshadepath{myshadingE}{0}
  \pgfusepath{stroke}
  \pgfpathrectangle{\pgfpoint{5cm}{0cm}}{\pgfpoint{2cm}{2cm}}
  \pgfshadepath{myshadingE}{45}
  \pgfusepath{stroke}
  \pgfpathcircle{\pgfpoint{9cm}{1cm}}{1cm}
  \pgfshadepath{myshadingE}{45}
  \pgfusepath{stroke}
\end{pgfpicture}

\pgfdeclareverticalshading{myshadingF}{100bp}
  {color(0bp)=(red); color(25bp)=(green); color(75bp)=(blue); color(100bp)=(black)}
\begin{tikzpicture}
  \draw (50bp,50bp) node {\pgfuseshading{myshadingF}};
  \draw[white,thick] (25bp,25bp) rectangle (75bp,75bp);
  \draw (50bp,0bp) node[below] {first two applications};
  \begin{scope}[xshift=5cm]
    \draw (50bp,50bp) node{\pgfuseshading{myshadingF}};
    \draw[rotate around={45:(50bp,50bp)},white,thick] (25bp,25bp) rectangle (75bp,75bp);
    \draw (50bp,0bp) node[below] {third application};
  \end{scope}
  \begin{scope}[xshift=10cm]
    \draw (50bp,50bp) node{\pgfuseshading{myshadingF}};
    \draw[white,thick] (50bp,50bp) circle (25bp);
    \draw (50bp,0bp) node[below] {fourth application};
  \end{scope}
\end{tikzpicture}

\pgfdeclareradialshading{ballshading}{\pgfpoint{-10bp}{10bp}}
  {color(0bp)=(red!15!white); color(9bp)=(red!75!white);
   color(18bp)=(red!70!black); color(25bp)=(red!50!black); color(50bp)=(black)}
\pgfuseshading{ballshading}
\hskip 1cm
\begin{pgfpicture}
  \pgfpathrectangle{\pgfpointorigin}{\pgfpoint{1cm}{1cm}}
  \pgfshadepath{ballshading}{0}
  \pgfusepath{}
  \pgfpathcircle{\pgfpoint{3cm}{0cm}}{1cm}
  \pgfshadepath{ballshading}{0}
  \pgfusepath{}
  \pgfpathcircle{\pgfpoint{6cm}{0cm}}{1cm}
  \pgfshadepath{ballshading}{45}
  \pgfusepath{}
\end{pgfpicture}

\pgfdeclareverticalshading{myshadingG}{100bp}
  {color(0bp)=(red); color(25bp)=(green); color(75bp)=(blue); color(100bp)=(black)}
\begin{pgfpicture}
  \pgfpathrectangle{\pgfpointorigin}{\pgfpoint{2cm}{1cm}}
  \pgfshadepath{myshadingG}{0}
  \pgfusepath{stroke}
  \pgfpathrectangle{\pgfpoint{3cm}{0cm}}{\pgfpoint{2cm}{1cm}}
  \pgfshadepath{myshadingG}{90}
  \pgfusepath{stroke}
  \pgfpathrectangle{\pgfpoint{6cm}{0cm}}{\pgfpoint{2cm}{1cm}}
  \pgfshadepath{myshadingG}{45}
  \pgfusepath{stroke}
\end{pgfpicture}

\pgfdeclareverticalshading{rainbow}{100bp}
  {color(0bp)=(red); color(25bp)=(red); color(35bp)=(yellow);
   color(45bp)=(green); color(55bp)=(cyan); color(65bp)=(blue);
   color(75bp)=(violet); color(100bp)=(violet)}
\begin{tikzpicture}[shading=rainbow]
  \shade (0,0) rectangle node[white] {\textsc{pride}} (2,1);
  \shade[shading angle=90] (3,0) rectangle +(1,2);
\end{tikzpicture}

\end{document}
